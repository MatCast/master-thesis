\chapter{Methodology}\label{chap2}
\thispagestyle{plain}

%This chapter provides a detailed description of the methodology. It is
%sometimes called Experimental Section. Depending on the subject it is a
%``synonym'', e.g., for Theoretical Section, Computational Methods, Model
%Description and Setup, Field Work, and so on. Hence, this chapter contains a
%description of \emph{what has been done} in order to address the scientific
%question raised in the chapter Introduction. However, it does \emph{not} contain
%the results! 


% ==== SECTION 1 ===============================================================
%\section{Experimental Set-up}\label{2sec:1}
%Depending on the topic of the science thesis, this chapter may contain a
%description of the experimental set-up, the field experiment,
%datasets, instruments, measurement procedures, analysis techniques, calibration
%and quality control, and other things. In case of a modeling study it may
%contain the formulation and derivation of model equations, the formulation of
%initial and boundary conditions, the data used to drive and validate the model,
%an overview of the model set-up (e.g., parameter set-up), modifications of the
%``original'' model code, a description of relevant parameterizations,
%a theoretical background needed for the interpretation of model results.


% ==== SECTION 2 ===============================================================
\section{Data Sets}\label{glathida}
This section gives an overview of the data used in the analysis done in this thesis. The basics data to run all the analysis have been the Glacier Thickness Database (GlaThiDa), the Randolph Glacier Inventory (RGI), and different digital elevation models (DEMs) which are dependent on the specific glacier. Depending on the region in fact some digital elevation models are a better fit to each glaciers as all of the freely available ones have missing data. ``The best'' digital elevation model has been automatically chosen by the Open Global Glacier Model (OGGM). Through most of the analysis SRTM v4 (\citet{SRTM}) which is a 90m grid resolution DEM has been used. Other DEM might have been used to link the GlaThiDa with RGI (see section \ref{GlaRGI})

\subsection{GlaThiDa}
The glacier Thickness Database (\citet{GlaThiDa2014}) is a set of data which attempts to put together all the available ice thickness measurements from glaciers and ice caps all around the globe. The database is composed of three different tables. The first one contains an overview of the database, the second one thickness data from maps or digital elevation models (DEM), and the third one, the one used in this thesis, contains the actual point measurements with the thickness data. The measurements were taken with different methods such as terrestrial and airborne radio-echo sounding, ground penetrating radar, direct drilling and other methods. The first database was released in 2014. In 2016 a second version 2.0 and its correction 2.1 were released, and in 2019 version 3.0 and 3.01 were finally released. All the analysis in this work have been done using version 2.1 as version 3.0 was released after a big part of the analysis was already done. According to the \href{https://github.com/ezwelty/glathida/blob/master/CHANGELOG.md}{change log} however most of the changes have been done to the structure of the data base. There were however some measurements additions to the database. The addition of some measurements for some glaciers in Switzerland in particular would be relevant for this thesis, given that most of the analysis done in this work has been conducted over the alpine glaciers. 
%Use subsections to structure your thesis. The first and second component of the
%momentum equation is shown in equation (\ref{2equ:1}) and (\ref{2equ:2}),
%respectively. Together with (\ref{2equ:3}) they form the set of shallow-water
%equations implemented in a numerical model.

\subsection{RGI}
The Randolph Glacier Inventory (RGI) (\citet{RGI2014}) is a global database of outlines of glaciers, excluding ice sheets. The inventory has been compiled from satellite imagery collected from 1999 onward. Most of the outlines don't express a specific picture in time of the glacier but are a compound of different images of each glacier due to images obstructions such as cloud covers or satellite orbits. The first version of this inventory was released in 2012 and the latest version, RGI Version 6.0 was released in 2017. In this latest release the database comprises more than 220,000 glacier outlines which are divided in 19 regions which cover all areas in the world with glaciers. This is the version used for all the analysis in this thesis. \todo{Add map of all the glaciers?}

\subsection{Linking GlaThiDa and RGI}\label{GlaRGI}
The GlaThiDa comes with a table containing the thickness measurements observations. This table comes with the thickness value of the observation, its latitude and longitude, and other variables such as the date of the observation, the name of the glacier, the thickness uncertainty and others. Aside from the measurements GPS coordinates and the thickness value, some fields like, the name of the glacier and the thickness uncertainty, are often left empty. This creates the problem of having to link each observation with a glacier in the RGI database. To do so a script has been created which determines whether an observation is located inside a glacier outline: the RGI database comes with closed outlines of the glaciers in geographic coordinates referenced to the WSG84 (also known as ESPG:4326) as shape-files. The python library \href{https://github.com/Toblerity/Shapely}{shapely} has been used to transform the observation Latitude and Longitude to the WSG84 projection system and to check whether each of these point was lying inside the glacier outlines (including the boundaries of the outline).
\todo{Add graphic of a glacier with the thickness points. I couldn't find a script with it so far.}

\subsection{Some statistics about GlaThiDa}
After linking the two databases it is interesting to learn about some statistics of the GlaThiDa. In order to create these statistics The Open Global Glacier Model (OGGM) (\citet{OGGM2019}, \href{https://github.com/OGGM/oggm}{Open Source Code}) has been used to add a digital elevation model to the glaciers in the RGI with thickness entries in the GlaThiDa (see \ref{GlaRGI}).

\subsubsection{Glaciers distribution and types}
There are 820370 entries (thickness measurements) in the GlaThiDa version 2.1. After assigning each of them to one of the 215,547 RGI glaciers 771 of those glaciers have thickness observations associated with them. This is 0.36\% of all the glaciers in th RGI. Out of the 820370 initial entries 27882, 3.4\% resulted outside of any glacier outlines defined in the RGI. Some reason for this could be: a slightly wrong GPS coordinate collected in the GlaThiDa measurement; observations taken outside of the glacier by mistake or intentionally, to make sure that all the glacier was covered; a different shape of the glacier at the moment when the observation was taken, compared to the moment when the RGI was compiled; a wrong assessment of the glacier shape in the RGI.
\begin{figure}\label{glathidamap} 
	\centering 
	\includegraphics[width=1.0\textwidth]{./figures/GlaThiDa_map.png}
	\caption{Global map of the distribution of glaciers with thickness observations entries in the GlaThiDa database version 2.1. Black outlines are the RGI regions.}
\end{figure}

Most of the glaciers with thickness observation are in the Greenland Periphery region and in the Artic Canada North region. If we compare the glaciers with measurements with the number of glaciers present in each specific region though, we see that Arctic Canada North is the best represented region with around 5\% of the glaciers having at least one thickness measurement point. The region with most glaciers, Central Asia, has a very low number of glaciers represented in the GlaThiDa database. This is probably due to the difficulty in getting measurements for glaciers as such high altitudes at those present in the Himalaya.


\begin{figure}[!tp]
	\centering		  
	\includegraphics[width=1.\textwidth]{figures/Observations_per_region.pdf}
	\caption{On the left side the number of glaciers in the RGI with thickness observations in the GlaThiDa are shown per RGI region. On the right side the percentage of glaciers for each region with thickness observations is shown instead.}
	\label{fig:glareg}
\end{figure}

More than two third of the RGI entities with GlaThiDa measurements are glaciers while the rest are ice caps. Land terminating glaciers and ice caps are the vast majority.

\subsubsection{Measurements Distribution per Glacier}
Not all the glaciers with measurements are perfectly covered over the whole glacier area. In fact around 42\% of the glaciers have less than 100 thickness observations. Given that some glaciers can extend for over $100 km^2$, it’s clear that some of them are very poorly covered.
\begin{figure}[!tp]
	\centering		  
	\includegraphics[width=1.\textwidth]{figures/Observations_per_skm.pdf}
	\caption{On the Left: distribution of glaciers with less than 100 observations per squared kilometer; there is a great number of glaciers with less than 20 thickness observations per squared kilometer. On the right: distribution of glaciers according to the percentage of their altitude bands covered.}
	\label{fig:glaobs}
\end{figure}

More than 91\% of all the glaciers (represented in the figure above) have less than 100 thickness measurements per squared kilometer. Almost 44\% of all the glaciers with thickness observations have less than 5 observations per squared kilometer (see Fig. \ref{fig:glaobs}).

OGGM was used to generate a gridded map of each glacier represented in the GlaThiDa. With this map one can divide each glacier in $100m$ altitude bands to check how many of those bands contain at least one thickness measurement.
Almost half of the glaciers have observations in at least half of the 100m altitude bands (see Fig. \ref{fig:glaobs}). Glaciers with few observations but well distributed over their length can be very useful for understanding glacier thickness patterns and for model validation.

\subsubsection{Survey Year}
Measurements were taken between 1977 and 2015, but most of them after 2005 (see Fig. \ref{fig:glayears}). The spread in the dates of the campaigns for taking the observations could potentially create problems when trying to compare glaciers with each other to find similar patterns, but also when using the data for model validation.
Most glacier models make in fact use of digital elevation models to setup the model run. Any digital elevation model is compiled with data available at the time of its construction. The time difference between the date of data assimilation for the digital elevation model, and the date of the survey for the thickness observation, could be in the order of several years. This would create an error when validating the model.
\begin{figure}[!tp]
	\centering		  
	\includegraphics[width=1.\textwidth]{figures/Observations_per_year.pdf}
	\caption{Distribution of glaciers with observations in the GlaThiDa according to the year when the measurements for each glacier were taken.}
	\label{fig:glayears}
\end{figure}


\section{Choosing Features}\label{features}

\subsection{Putting together the features}
How did we get the features for training: OGGM.

\subsection{Which features did we choose}
The features we chose and why.

\section{Machine Learning Algorithms}\label{ML}
Machine learning is a mathematical method which enables computers to perform a specific task without giving it the specific instructions on how to do it. This is achieved using algorithms to build a model from data usually called the ``training data''. This data are given to the algorithm which uses an iterative process to create the final model. 

The idea behind it is to emulate the learning process of a human-being in order to make predictions about new data. If for example one wanted to teach a kid how to recognize a cat, the simpler approach would be to show it cats or pictures of them, instead of trying to explain it what are the key attributes which characterize cats. In the same way machine learning algorithm are used to make predictions without giving them instructions (explaining the key characteristics) about how to do the predictions, but rather by feeding them examples.

The branch of machine learning used in this thesis is called ``supervised learning'' which builds models learning from data which contain the input variables and the desired output value, often refereed to as labels.
The machine learning algorithm is then used to find the target function $f$ which better maps input vector $X$ to an output one $Y$.
\begin{equation}
Y = f(X)
\end{equation}
The way this function is computed by the algorithm depends on the specific one chosen to achieve the desired result.

In this thesis three different machine learning algorithm have been used to train models which estimate the ice thickness of glaciers using the GlaThiDa database as the desired output ($Y$) for the ice thickness, and using digital elevation models and some basic physics assumptions as the input values ($X$), to obtain the ice thickness distribution of the glacier.

The three algorithm used are:
\begin{itemize}
	\item Linear Regression
	\item Random Forest Regression
	\item Support Vector Regression (SVR)
\end{itemize}

The algorithms have been applied to the data using the python package \href{https://scikit-learn.org/}{scikit learn}.
%The way the algorithm achieves this without instructions from the programmer is with an iterative process which uses the training data to make a first prediction, compare the prediction with the the desired output value, 
\subsection{Linear Regression}
Linear Regression is probably one of the simplest machine learning algorithm and also one of the most used (even though many people probably don't know it is a machine learning algorithm).

Given a set of input data $\bm{X} = \{x_{i1},\ldots ,x_{ip}\}_{i=1}^{n}$ and a corresponding output $\mathbf{y} = \{y_{1},\ldots ,y_{n}\}$, the algorithm finds the best set of $\bm{\beta} = \{\beta_{i1},\ldots ,\beta_{ip}\}_{i=1}^{n}$ and $\bm{\varepsilon} = \{\varepsilon_{1},\ldots ,\varepsilon_{n}\}$ such that:
\begin{equation}
\mathbf {y} = \bm{\varepsilon} + \bm{\beta}\bm{X}
\end{equation}
In order to find $\bm{\varepsilon}$ and $\bm{\beta}$ which best predict the desired $\mathbf {y}$, one of the most common approaches is to minimize the loss function
\begin{equation}
\frac{1}{n}\sum_{1}^{n}(pred_i-y_i)^2
\end{equation}
where $pred_i$ is the value of the $i$-th prediction generated by the model.

This algorithm is clearly only useful for cases when the output variables linearly depend on the input ones. This is however often not the case thus requiring most sophisticated models to compute the desired output values. It can be helpful however to compare this method to the more sophisticated ones to have a basic reference on their performances.

\subsection{Random Forest}
\begin{figure}[!tp]
	\centering		  
	\includegraphics[width=1.\textwidth]{figures/decision_tree.png}
	\caption{Representation of a decision tree model. The tree is pictured upside-down with the root on the top and the branches and leaves on the bottom.}
	\label{fig:tree}
\end{figure}
Random Forest is an algorithm which uses ensemble learning to extend and improve the performances of decision trees algorithm. The method was first proposed by \citet{RandomHo1995}. A decision tree is an algorithm used to predict the desired value from a set of data. It works by splitting the data-set into a number of branches by imposing conditions on each feature. Every split leads to a subset of the data-set which can be split further to better achieve the desired outcome(see \ref{fig:tree}). In order to choose how to split the data on each node the algorithm calculates the sum of the squared residual for splitting the data-set on a particular condition and chooses the condition which minimizes this sum. Let $\bm{X} = \{x_{i1},\ldots ,x_{ip}\}_{i=1}^{n}$ be the $n$-th input values $p$ the different attributes  our data-set and $\mathbf{y} = \{y_{1},\ldots ,y_{n}\}$ the output values (the values we want to predict). Let's assume for simplicity that we are splitting the data based on a condition on the first attribute. The first split in the tree divides the data-set into two subsets $S_1$ with $l$ data points and $S_2$ with $n-l$ data points. The subsets are defined so that the following is minimized:
\begin{equation}\label{eq:treeres}
\sum_{i \in S_1}(\overline{y_1}-y_i)^2 + \sum_{j \in S_2}(\overline{y_2}-y_j)^2
\end{equation}
where  $\overline{y_1} = \frac{1}{l}\sum_{i \in S_1}y_i$ and $\overline{y_2} = \frac{1}{n-l}\sum_{j \in S_2}y_j$. The method can then be repeated to create further branches and a deeper tree. In order to choose which feature to use as a condition for the split, \ref{eq:treeres} is calculated for each feature and the one resulting in the minimum value draws the condition for the split on the node. The algorithm usually stops when the minimum arbitrary number of observations for each subset is reached.

The problem with decision trees is that they are inaccurate for predictions with data outside the ones used to create them (\cite{hastie01statisticallearning}).
To solve this problem Random Forest algorithm relay on an ensemble of decision trees to make the predictions. Each tree in the ensemble (forest) is constructed using the same number of input values $n$, but randomly selecting the input values with replacement: each input value can be chosen more then once to be part of the sample constructing the three, leaving out some of the input values from this sample. In addition to this, each tree is only limited to a subset of features to choose from to split the data at each node. After each tree is trained the prediction is made by averaging the predictions from al the trees in the forest. Additionally an estimate of the uncertainty of the prediction can be calculated from the standard deviation of the predictions from all the trees.



\subsection{Support Vector Regression}
Support Vector Regression (SVR) was first presented as regression extension of the support vector machine algorithm (SVM) by \citet{SVR1997}. Support vector machine is a supervised machine learning algorithm for classification (i.e. the desired output has binary values) first proposed by \citet{SVM1964}.

Let's assume a training data-set of $n$ points ${({\mathbf {x}}_{1},y_{1}),\ldots ,({\mathbf {x}}_{n},y_{n}),}$, where $y_{i}$ are either $1$ or -$1$, each indicating a different class, the value we want to predict, for the points ${\mathbf {x}}_{i}$. Each ${\mathbf {x}_{i}}$ is a $p$-dimensional real vector. We want to find the decision boundary that divides the group of points $\mathbf {x}_{i}$ for which $y_{i}=1$ from the group of those for which $y_{i}=-1$. This boundary is defined in such a way that the distance between the hyperplane and the closest ${\mathbf {x}}_{i}$ from either group is maximized.
An hyperplane is defined as the points ${\mathbf {x}}$ satisfying 
\begin{equation}\label{eq:svm}
\mathbf {w}\cdot \mathbf {x}-b=0
\end{equation}

where $\mathbf {w}$ is the normal vector to the plane. $b$ determines the offset of the hyperplane from the origin.

If the training data were linearly separable two parallel hyperplanes must exist separating the two classes of data points such that the distance between the two is maximized. The equations for these planes are respectively:

\begin{align}
\label{eq:planes1}
	\begin{split}
	\mathbf {w}\cdot \mathbf {x}-b = 1, \\
	\mathbf {w}\cdot \mathbf {x}-b = -1
	\end{split}
\end{align}

These two hyperplanes would determine the margin between the two groups. The data points ${\mathbf {x}}_{i}$ lying nearest to these two hyperplanes are called the support vectors. The hyperplane lying exactly in the middle of the two is called decision boundary. This hyperplane separates the two groups of data. If one wanted to label a new data point not used to train the model, this point would be classified depending on which side of the decision boundary it would lie on. This is called a hard margin.

\begin{figure}[!tp]
	\centering		  
	\includegraphics[width=1.\textwidth]{figures/SVM.png}
	\caption{Representation of support vector machine in a 2-dimensional space. On the left a model is trained with the creation of an hard-margin which correctly classifies each sample in the training data-set. On the right a soft-margin is selected which allows a number of miss-classified points to lie inside the margin.}
	\label{fig:svm}
\end{figure}

The distance between the two plains is ${\frac {2}{\|{\mathbf {w}}\|}}$. In order to maximize this distance we then need to minimize $\|{\mathbf {w}}\|$, with the constrain

\begin{equation}\label{eq:svmconstrain}
y_{i}({\mathbf {w}}\cdot {\mathbf {x}}_{i}-b)\geq 1
\end{equation}
for all $i=1,\ldots ,n$ given by the fact that each point must lie outside the margin (see Eq. \ref{eq:svrconstrain}).

This algorithm works well for linearly separable data-sets. If however the data were non linearly separable, or if outliers or miss-classifications were included in the data-set, as in most real  world cases, one would instead choose a so called "soft-margin" which would work exactly like the hard margin but with allowance for a number of miss-classified data points (see Fig. \ref{fig:svm})

To extend this concept for a regression case, i.e. a case where the output values $y_{i}$ are continuous, we can use the same concept with but with the objective of finding a function $f(x)$ which deviates from each $y_{i}$ by a value not greater than $\varepsilon$ and is as flat as possible.

\begin{equation}\label{eq:svr}
f(x) = \mathbf {w}\cdot \mathbf {x}-b
\end{equation}

This would mean to minimize $\|{\mathbf {w}}\|$ subject to the constrain:

\begin{equation}\label{eq:svrconstrain}
|y_{i} - {\mathbf {w}}\cdot {\mathbf {x}}_{i}+b|\leq \varepsilon
\end{equation}
for all $i=1,\ldots ,n$.

To extend the algorithm to non linear cases SVM and SVR apply the so called "kernel trick". The algorithm is applied in a transformed higher dimensional space. The resulting boundary hyperplane is linear in the transformed space but might be non linear in the original space.
There are different ways, called kernels, to transform the space to a higher dimensional one. Some of the most common ones, and the ones available in the package \href{https://scikit-learn.org/}{scikit learn} used for the analysis in this thesis, are:
\begin{itemize}
	\item Polynomial
	\item Radial Basis Function
	\item Sigmoid
\end{itemize} 

The one chosen for the analysis in this thesis is the Radial Basis Function kernel (rbf) which transforms the space like so:
\begin{equation}\label{eq:rbf}
k({\mathbf {x_{i}}},{\mathbf {x_{j}}})=\exp(-\gamma \|\mathbf {x_{i}}-\mathbf {x_{j}}\|^{2})
\end{equation}
for $\gamma >0$.
The algorithm basically adds dimensions which are dependent on the distance between the data points. Points with higher distances are less influenced by each other because of the exponential decay which relates the distances.

The kernel trick and is what makes support vector regression so powerful because it allows the algorithm to learn from non linear relationship between the data.

\subsection{Tuning parameters}
How did we decide which parameters to use (maybe we can just leave it for the appendix)

\section{Training method}\label{training}
use sklearn train\_test\_split 20 times

\section{Scoring method}\label{scoring}
metrics to compare the goodness of models

\section{Features Importance}\label{featuresimp}
what is feature importance.

\subsection{Shuffle}
Shuffle

\subsection{partial dependence plot}
partial dependence plot 
%\subsubsection{Subsubsection}
%You can also use ``subsubsections''. However, they do not carry a separate
%heading number and they do not appear in the Table of Contents.

%\subsection{Equation}
%As an example for the \verb|equation| environment, I show the equations used in
%the numerical shallow-water model (SWM) developed by
%\citet{scha93Aag,scha93Bag}:
%% ---- equation 1:
%\begin{equation}
%\frac{D\hat{u}}{D\hat{t}}+\frac{\partial(\hat{h}+\hat{H})}
%                               {\partial\hat{x}}=0,
%\label{2equ:1}
%\end{equation}
%% ---- equation 2:
%\begin{equation}
%\frac{D\hat{v}}{D\hat{t}}+\frac{\partial(\hat{h}+\hat{H})}
%                               {\partial\hat{y}}=0,
%\label{2equ:2}
%\end{equation}
%% ---- equation 3:
%\begin{equation}
%\frac{\partial\hat{H}}{\partial\hat{t}}+\frac{\partial(\hat{u}\hat{H})}
%                                             {\partial\hat{x}}
%                                       +\frac{\partial(\hat{v}\hat{H})}
%                                             {\partial\hat{y}}
%                                                =0,
%\label{2equ:3}
%\end{equation}
%with the non-dimensional variables (henceforth generally labelled with hats)
%$\hat{u}$ and $\hat{v}$ as the two horizontal velocity components, $\hat{H}$ and
%$\hat{h}$ as fluid layer depth and terrain height, respectively,
%$\hat{Z}=\hat{h}+\hat{H}$ as fluid layer height, and $\hat{t}$ as time.
%Equations~(\ref{2equ:1})--(\ref{2equ:3}) are non-dimensionalized with the
%following scales: a typical length $L$ for the horizontal length scale, the
%initial far-upstream depth of the fluid layer $H_{\infty}$ (with $h_{\infty}=0$)
%for the vertical length scale, the phase speed of linear gravity waves
%$\sqrt{g^*H_{\infty}}$ for the velocity scale, and the time scale
%$L/\sqrt{g^*H_{\infty}}$.
