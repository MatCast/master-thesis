\chapter*{Abstract}
\addcontentsline{toc}{chapter}{Abstract}
\thispagestyle{plain}
Melting glaciers play an extremely important role in the context of climate change. Correctly estimating their volumes is essential to accurately predict changes such as sea level rise. To compute the volume of a glacier its ice thickness distribution is needed. Not many glacier ice thickness measurements are available. Modeling glaciers becomes necessary to estimate their ice thickness distribution. Recently the Ice Thickness Models Intercomparison eXperiment (ITMIX) has looked into the accuracy of 17 global glacier models. 15 of those models numerically model physical processes in order to estimate the ice thickness distribution of the glaciers. Statistical glacier models based on ice thickness observations have not yet been used to compute the ice thickness of glaciers. The recent creation of the Glacier Thickness Database (GlaThiDa), and the advancement in the field of machine learning, open up to the possibility of a glacier model based on the learning ability of a machine learning algorithm.

In this thesis three different machine learning algorithm have been trained using ice thickness observations in the GlaThiDa for glaciers in the alps. Linear regression, random forest regression and support vector regression have been used to create models to estimate glacier ice thickness. Part of the ice thickness observations have been used to train the models and part to test their accuracy. With this method 20 different samples have been created to train and test the models. The highest $R^2$ coefficient has been registered by the support vector machine with an average $R^2=0.55$ across the different samples. 

The algorithms have also been trained using all the ice thickness observations available for glaciers in the alps. The highest $R^2=0.57$ has been registered for the random forest regression. The results obtained from these models have been used to estimate the total volume of glaciers in the alps. The results have been compared to the findings from the third ITMIX experiment. The linear regression model predicted a total volume 9\% larger than the ITMIX. The random forest regression a volume 22\% lower and the support vector machine a volume 25\% lower. The last two models have been found unable to correctly estimate ice thicknesses for large glaciers in the alps.   
%The abstract is a short summary of the thesis. It announces in
%a brief and concise way the scientific goals, methods, and most important
%results. The chapter ``conclusions'' is not equivalent to the abstract!
%Nevertheless, the abstract may contain concluding remarks. The abstract
%should not be discursive. Hence, it cannot summarize all aspects of the thesis
%in very detail. Nothing should appear in an abstract that is not also
%covered in the body of the thesis itself. Hence, the abstract should be the
%last part of the thesis to be compiled by the author.
%
%A good abstract has the following properties: \emph{Comprehensive:} All major
%parts of the main text must also appear in the abstract. \emph{Precise:}
%Results, interpretations, and opinions must not differ from the ones in the main
%text. Avoid even subtle shifts in emphasis. \emph{Objective:} It may contain
%evaluative components, but it must not seem judgemental, even if the thesis
%topic raises controversial issues. \emph{Concise:} It should only contain the
%most important results. It should not exceed 300--500 words or about one page.
%\emph{Intelligible:} It should only contain widely-used terms. It should
%not contain equations and citations. Try to avoid symbols and acronyms (or at
%least explain them). \emph{Informative:} The reader should be able to quickly
%evaluate, whether or not the thesis is relevant for his/her work.
%
%An Example: The objective was to determine whether \dots (\emph{question/goal}).
%For this purpose, \dots was \dots (\emph{methodology}). It was found that \dots
%(\emph{results}). The results demonstrate that \dots (\emph{answer}).
