\chapter*{Abstract}
\addcontentsline{toc}{chapter}{Abstract}
\thispagestyle{plain}

%The abstract is a short summary of the thesis. It announces in
%a brief and concise way the scientific goals, methods, and most important
%results. The chapter ``conclusions'' is not equivalent to the abstract!
%Nevertheless, the abstract may contain concluding remarks. The abstract
%should not be discursive. Hence, it cannot summarize all aspects of the thesis
%in very detail. Nothing should appear in an abstract that is not also
%covered in the body of the thesis itself. Hence, the abstract should be the
%last part of the thesis to be compiled by the author.
%
%A good abstract has the following properties: \emph{Comprehensive:} All major
%parts of the main text must also appear in the abstract. \emph{Precise:}
%Results, interpretations, and opinions must not differ from the ones in the main
%text. Avoid even subtle shifts in emphasis. \emph{Objective:} It may contain
%evaluative components, but it must not seem judgemental, even if the thesis
%topic raises controversial issues. \emph{Concise:} It should only contain the
%most important results. It should not exceed 300--500 words or about one page.
%\emph{Intelligible:} It should only contain widely-used terms. It should
%not contain equations and citations. Try to avoid symbols and acronyms (or at
%least explain them). \emph{Informative:} The reader should be able to quickly
%evaluate, whether or not the thesis is relevant for his/her work.
%
%An Example: The objective was to determine whether \dots (\emph{question/goal}).
%For this purpose, \dots was \dots (\emph{methodology}). It was found that \dots
%(\emph{results}). The results demonstrate that \dots (\emph{answer}).
