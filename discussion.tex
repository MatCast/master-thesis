\chapter{Discussion}\label{disc}
\thispagestyle{plain}

The discussion is the interpretation and evaluation of the results. It is a
comparison of your results with previous findings. It provides the answer to the
scientific questions raised in the introduction. It is the ``nerve center'' of a
thesis, whereas the chapter Results may be seen as the ``heart''.

Clearly separate between your own contributions and those of others. Provide
rigorous citations of appropriate sources! Explicitly refer to specific results
presented earlier. A certain amount of repetition is necessary. For
example, the results presented in \ref{3sec:2} suggest that \dots. Order
discussion items not chronologically but rather logically.

The chapter Results answers the question: \emph{What} has been
found? (Facts). The chapter Discussion answers the question: \emph{How} has the
result to be interpreted? (Opinion).

The most important message should appear in the first paragraph. The answer to
the key question may appear in the first sentence: e.g., did your original idea
work, or didn't it? The following questions may be answered in the discussion
section:
\begin{itemize}
\item Why is the presented method simpler, better, more reliable than previous
ones?
\item What are its strengths and its limitations?
\item How significant are the results?
\item How trustworthy are the observations?
\item Under which precondition/assumption and for which region are the
results/method valid?
\item Can the results be easily transferred to other regions or fields?
\end{itemize}
