\chapter{Conclusions}\label{concl}
\thispagestyle{plain}

In this thesis machine learning algorithm have been used to create models aimed at predicting glaciers ice thickness.

In order to do so the GlaThiDa, a database with thickness observations of glaciers all over the world has been linked to the RGI, a database of glaciers outlines which include over 215.000 glaciers which should include all glaciers and ice caps in the globe. Of all the observations in GlaThiDa 96.6\% have been found to part of one of the glaciers in the RGI representing a total of 771 glaciers worldwide. Many of these glaciers only present few observations per squared kilometer and some regions don't have glaciers with observations entries at all.

In order to create a data set of input data to train the machine learning models the Open Global Glacier Model has been used. This model provided the tools to merge together the relevant digital elevation models for each glacier and compute gridded geometrical feature such as: the topography, the distance from the border of the glacier, the slope angle and the altitude base mass balance.

With this data 3 different machine learning models, linear regression, random forest regression and support vector regression, have been trained using only observations available for the alpine region. To avoid overfitting the models and to better be able to asses the performances of each model the each model has been trained with 20 different sub-samples each containing 75\% of data points available. The $R^2$ coefficient, also called score, calculated over the resulting 25\% of the data, has been chosen as a value to asses the performances of each model. The best performing model on average was the support vector regression with an average $R^2 = 0.55$. The score spread between the models trained with the different sub-samples has been found particularly high with much lower scores for some specific sub-samples. This is true for the three different machine learning algorithm used as all of them showed drops in score for samples 5, 14 and 19. This behavior seems explainable by the fact that the sub-sample left out for training in these 3 cases was particularly hard to predict.
The model achieving the highest score when training them over the whole sample has been the random forest regression with a score of 0.57.

After the models have been trained the relevance of each feature used for training when making a prediction has been analyzed. For the three models the mass balance and the slope angle have emerged as the most determining features for making predictions. The mass balance was the most relevant feature for both the random forest and support vector regression but for the linear regression it was the slope angle.

The predictions made by the models have then been extended for all the glaciers in the alps to compute their volumes. This result has been compared to the one obtained by \citet{Farinotti2019} in the 3rd ITMIX. All the machine learning models have been found to predict very different volumes compared to the ones predicted in the ITMIX. The linear regression is the only model which predicted a total ice volume for glaciers in the alps larger than the one from ITMIX by over 9\%. The random forest and support vector machine both predicted a total volume over respectively 22\% and 25\% lower than ITMIX. These two models in fact seem to underestimate the ice thickness for large glacier leading to low volume values for those glaciers and hence for the whole region. Larger glacier are essential in estimating the total volume of ice but are also outliers in the per-volume distribution of glaciers in the alps. The machine learning algorithm seem then unfit to make predictions about these outliers. Part of the reason could be the fact that the models were not able to learn the patterns of these large glaciers do to the fact that they had no information about them in the training data-set. 

After all the machine learning algorithm have not been found very accurate in estimating the ice thickness at least using the features chosen as input values. This is especially true for large glaciers for which the models seem unable to replicate accurate ice thicknesses.
Some more research could be done using different models and adding more features to the input data.
The results obtained for the alps could be different for other regions especially if those regions had ice thickness observations covering a wider spread of glaciers dimensions (small, mid an large glaciers equally).
Given the importance of large glaciers in computing ice volumes having more ice thickness measurements for those glaciers could improve the curacy of the models at predicting their thicknesses and with this the predictions for the ice volumes. 

\begin{itemize}
	\item link glathida rgi
	\item generate attributes using oggm
	\item train the models with sub-samples
	\item analyze feature importance
	\item alpine volume
\end{itemize}

%This chapter contains consequences that derive from your results. It may also
%contain speculations. It may provide suggestions for future studies. Hence, the
%conclusions may provide an outlook and list open questions. Sometimes
%this chapter is part of the discussion. In such a case, the chapter reads
%``Discussion and Conclusions''.

To be decided
