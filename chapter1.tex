\chapter{Introduction}\label{chap1}
\thispagestyle{plain}

% ====SECTION 1 ================================================================
\section{Motivation}\label{motivation}
Estimating the volume of glaciers is becoming an increasingly important topic, due to the role this plays in climate change. Melting glaciers have in fact an impact in sea level rise (\citet{Zemp2017}), availability of fresh drinking water (\citet{Kaser2010}) and even energy supplied by hydro electric power plants(\citet{Terrier2011}). A reliable estimate of glaciers volume is key in order to properly compute the entity of these and more problems.

Computing the volume of glaciers is a trivial task once in possession of the data needed for it: the glacier surface and its distributed ice thickness. These data however are not easily obtainable due to the amount of effort needed to collect them. A very big effort in this regard has been carried out in the creation of the Randolph Glacier Inventory (RGI) (\citet{RGI2014}), which is a data set containing a globally complete inventory of glaciers and their surface shapes. Data on ice thicknesses for the glaciers in the RGI however are much more lackluster (see \ref{featuresimp}). Of the over 200,000 glaciers included in RGI, only 771 have thickness observations, and in most of the cases not enough to have a complete thickness distribution along the glacier surface. These data are collected in the Glacier Thickness Database (GlaThiDa \citet{GlaThiDa2014}) which is the first global effort in collecting all the available ice thickness data in a single database.

To obtain an estimate of volume for glaciers lacking thickness measurements (the vast majority of the global glaciers), the only possible alternative is to model them, which is where a big effort from the scientific glaciology community has been and is being directed.
  
%The chapter Introduction leads the reader into the subject matter
%of the thesis. It is sometimes called
%Statement/Formulation/Definition/Presentation of the Problem. It may start with
%a so-called Motivation. It also contains the State of Knowledge or State
%of Research which is based on a literature survey (see section \ref{1sec:2}).
%Further, it contains the Scientific Questions and/or the Goals that are
%addressed in the main part of the thesis (see section \ref{1sec:3}). Finally,
%it provides an Outline of the science thesis (see end of section \ref{1sec:3}). 


% ==== SECTION 2 ===============================================================
\section{State of Research}\label{research}
In the years a number of different approaches have been used to model glaciers and estimate their volumes.
The most simple ones only attempt to compute these volumes from the relationship between the glacier surface and its volume as first stated by \citet{bahr1997}. Using the Buckingham Pi
theorem they found that the relationship between the surface area $S$ and the volume of a glacier $V$ is $V \propto S^{\gamma}$ and they estimated $\gamma=1.375$ for these glaciers and $\gamma=1.25$ for ice caps. Similar approaches also take into consideration other other parameters such as glacier length, width, elevation range (\cite{Grinsted2013}), and surface slope. These type of methods are called ``scaling approaches'' and they yield to a volume estimate for each glacier but not to the distributed ice thickness. Due to the small requirements for their implementations they are very effective at computing the global volume of glaciers.

A more sophisticated approach for calculating the volume of glaciers would be to numerically model the whole glacier and extrapolate the volume from the results of this simulation. Usually physical assumptions must be made to be able to reproduce the glacier bed from its surface ice distribution.
First steps in this direction were made by \citet{Nye1965} in the study of  idealized glaciers. He pointed out that the distributed ice thickness could be inferred by the glacier surface slope using estimates of basal shear stress. From there many others developed this idea such as \citet{haeberli1995} who made the assumption of a known constant basal shear stress, parameterized from the glacier elevation range to estimate basic glaciological characteristics for glaciers in the Alps. 

From these basic concepts, many more approaches have emerged in the years after, taking for example the mass conservation into account \cite{rasmussen_1988}, basal slipperiness together with bedrock topography \cite{Gudmundsson2001} and even more complex ones using forward models of ice flow \cite{vanPelt2013}. Also some non physical methods such as artificial neural networks \cite{Clarke2009} have been adopted all to tackle the problem of estimating glaciers ice thickness. All these achievements have been possible thanks to the advancement in research in the glaciological field and to the steadily increase in computational power and its availability. Models which could only be numerically computed for single glaciers started expanding their range of possibilities and \citet{Huss2012} released the first glacier model used to model all known glaciers and ice caps worldwide.

Since then many more global scale glacier models able to predict ice thickness distribution have been released. Meanwhile the GlaThiDa database was being compiled. This led to the first Ice Thickness Models Intercomparison eXperiment (ITMIX) (\cite{Farinotti2017}), which is the first attempt of putting together all those models and asses their accuracy on the base of known ice thickness observations available in the GlaThiDa as well as comparing the models performances with each other. 

The experiment focused on 17 glacier models able to predict glacier ice thickness on a global scale. The task of each model was to estimate the ice thickness distribution of at least some of 15 glaciers, 3 ice caps and 3 synthetically grown glaciers. The output of the models was compared with data available from the GlaThiDa and with the other models. A composite solution from all the participating models achieved a local ice thickness difference of the order of $10 \pm 24\%$ of the mean ice thickness. In general however the spread of the solutions obtained by the models often exceeded the local ice thickness, concluding that an ensemble approach could mitigate errors and would reduce the random nature of single models output. A third ITMIX (\cite{Farinotti2019}) has been developed using 5 of the models form the first ITMIX to create this ensemble model for better predicting ice thickness distributions of all glaciers in the world.  

The models participating in the first ITMIX were divided into categories depending on the different approaches used to model the ice thickness. The categories are: (1) those taking ice thickness inversion approaches as a minimization problem, (2) those based on mass conservation,  (3) approaches depending on a basal shear stress parametrization, (4) approaches based on surface velocities and (5) other approaches. 15 of the 17 explored models in the ITMIX, belong to the first 4 categories, which are all categories for methods relaying on physical assumptions to numerically model some of the physical processes happening in a glacier and predict the ice thickness from there. Only 2 models, belonging to category (5) are models not based on physical assumptions and methods, but rather on statistical methods. Of those 2 the one from \citet{Clarke2009} is based on artificial neural networks, a branch of machine learning. It uses the input data from artificially grown glaciers to make predictions about real glaciers ice thickness learning from the input data of the artificially grown ones.

Machine learning is a branch of computer science which uses different algorithms to give a computer the ability to learn from a set of samples. The concept was of artificial intelligence, which is the super field of machine learning, was first proposed by \citet{Turing1950}. Machine learning algorithm which are still used today however were already available before Turing's formulation. The basics principle behind all of the is to allow a computer program to generate a model capable of solving a task, without explicit instructions on how to solve the problem. In the case of the glacier thickness it would mean to give the algorithm a data sample and have the algorithm be able to predict ice thickness by learning from the data-sample. 
There are many different machine learning algorithm which have been developed since 1950 and, as mentioned, some which have been developed even before.

Although the algorithms have been around for a long period of time they only saw a wide spread and application starting from the late 1990 as computation power and resources largely improved. Since then however the field has had a huge development and many of its application have become part of our day to day life. Things such as voice recognition, chat bots, autonomous driving and online products suggestions are all a product of machine learning algorithm being developed and trained to solve this specific problems.

As mentioned above however machine learning has yet to make a big impact in the glaciological community and most of the models used in the experiment, and the ones which achieved the better accuracy, are physical based model. The model from \citet{Clarke2009} which uses artificial neural networks to predict the glacier ice thickness is has not been trained on real glacier data but rather on artificial generated ones; at the time of publication the GalThiDa had not yet been compiled and there were not enough ice thickness observations available to train a model from real glacier data.

The availability of ice thickness observations, the excitement behind machine learning in almost every possible field, and the lack of previous work done in this sense by the glaciological modeling community is what drives the motivation behind this thesis.
%Based on the literature survey, the writer draws a picture of the existing
%knowledge in a specific field and points to open questions. Hence, after this
%survey the Introduction will ultimately culminate in the formulation of specific
%scientific questions/goals addressed in the thesis (see section \ref{1sec:3}).

%To cite a certain source (e.g., a paper) use the citation commands
%\verb|\citet| and \verb|\citep| of the \verb|natbib|
%package.\footnote{
%\url{http://www.ctan.org/tex-archive/macros/latex/contrib/natbib/natnotes.pdf}}
%Together with the bibliography style \verb|ametsoc.bst|, which is
%included in the \LaTeX{} manuscript template for AMS journals\footnote{
%\url{http://www.ametsoc.org/PUBS/journals/AMS_Latex_V3.0.tar.gz}}, \verb|natbib|
%produces citations in the author-date format together with a list of
%references that fulfill the AMS citation standard.\footnote{
%\url{http://www.ametsoc.org/PUBS/journals/author_reference_guide.pdf}}
%
%As an example, you can cite papers like \citet{hann66Aag} and \citet{scha93Aag}
%which have to be specified in your BibTeX database file (in this case it is
%\verb|mybibfile.bib|). More than one article of the same author can be cited
%like here: \citet{hoin85Aag,hoin90Aag} studied foehn winds.
%
%You may want to split your review of the literature into several sections.
%Further, use paragraphs to structure your introduction. If you like to cite
%papers in brackets (\emph{passive citations}) you can do this
%as in the following sentence: Gap flows have been studied in the Strait
%of Gibraltar \citep{scor52Aag,dorm95Aag}, in the French Rh\^one Valley
%\citep{pett82Aag}, near Hokkaid\=o in Japan \citep{arak69Aag}, near
%Unimak Island in the Aleutian Chain \citep{pan-99Aag}, and in the Howe
%Sound of British Columbia \citep{jack94Aag,jack94Bag}. Citation of a
%Dissertation: The gap flow in the Wipp Valley has been studied by
%\citet{gohm03Aag}. Citation of a conference paper: \citet{gohm06Aag}
%investigated the boundary layer structure in the Inn Valley. Citation of an
%online document: The AMS provides a guideline for preparing citations and
%references \citep{ams-09Aag}.


% ==== SECTION 3 ===============================================================
\section{Goals and Outline}\label{goals}
In this thesis three different machine learning algorithms have been chosen to  create models able to predict glacier ice thickness. The algorithms have all been trained on the same data-set the GlaThiDa which contains ice thickness observations for glaciers worldwide.

To train the algorithm a set of attributes (features), mainly geometrical, have been generated using the Open Global Glacier Model (OGGM) \cite{OGGM2019}. This attributes are all computable by the only mean of a digital elevation model for the considered glacier considered and the outline of the glacier given by the RGI.

The accuracy of the models have been estimated comparing their predictions with the observations from the GlaThiDa as well as with themselves.

An analysis of the most influential attributes in making predictions has been carried out.

Finally the obtained models have been used to predict the volume of all the alpine glaciers and compare their results with those of the 3rd ITMIX \cite{Farinotti2019}.


This thesis then tries to establish:
\begin{itemize}
\item[(1)] How accurate are machine learning algorithm at predicting glaciers ice thickness.
\item[(2)] How 3 different machine learning algorithm compare with each other in estimating glaciers ice thickness.
\item[(3)] Using those algorithm to estimate the total volume of glaciers in the alps, how these models compare to the model from the \cite{Farinotti2019} elaborated in the 3rd ITMIX.
\end{itemize}

%The so-called SMART
%criteria\footnote{\url{http://en.wikipedia.org/wiki/SMART_criteria}} might be
%used as a guideline to define reasonable goals. Here, the acronym SMART
%describes the properties of ``good'' goals: \underline{s}pecific,
%\underline{m}easurable, \underline{a}ttainable, \underline{r}elevant,
%\underline{t}ime-bound.
%
%The introduction may also describe briefly the methodology chosen and the
%materials (e.g., data, instruments, etc.) used. However, a detailed description
%will follow in the main part (see chapter \ref{chap2}).
%
%Finally you should present an \emph{outline} of your science thesis. Explain
%what the reader will find in the following chapters. For example, chapter
%\ref{chap2} describes the methodology. The results are presented
%in chapter \ref{chap3}. A discussion is provided in chapter \ref{disc} and the
%conclusions are drawn in chapter \ref{concl}.
