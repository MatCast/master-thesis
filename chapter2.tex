\chapter{Methodology}\label{chap2}
\thispagestyle{plain}

%This chapter provides a detailed description of the methodology. It is
%sometimes called Experimental Section. Depending on the subject it is a
%``synonym'', e.g., for Theoretical Section, Computational Methods, Model
%Description and Setup, Field Work, and so on. Hence, this chapter contains a
%description of \emph{what has been done} in order to address the scientific
%question raised in the chapter Introduction. However, it does \emph{not} contain
%the results! 


% ==== SECTION 1 ===============================================================
%\section{Experimental Set-up}\label{2sec:1}
%Depending on the topic of the science thesis, this chapter may contain a
%description of the experimental set-up, the field experiment,
%datasets, instruments, measurement procedures, analysis techniques, calibration
%and quality control, and other things. In case of a modeling study it may
%contain the formulation and derivation of model equations, the formulation of
%initial and boundary conditions, the data used to drive and validate the model,
%an overview of the model set-up (e.g., parameter set-up), modifications of the
%``original'' model code, a description of relevant parameterizations,
%a theoretical background needed for the interpretation of model results.


% ==== SECTION 2 ===============================================================
\section{Data Sets}\label{glathida}
This section gives an overview of the data used in the analysis done in this thesis. The basics data to run all the analysis have been the Glacier Thickness Database (GlaThiDa), the Randolph Glacier Inventory (RGI), and different digital elevation models (DEMs) which are dependent on the specific glacier. Depending on the region in fact some digital elevation models are a better fit to each glaciers as all of the freely available ones have missing data. ``The best'' digital elevation model has been automatically chosen by the Open Global Glacier Model (OGGM). Through most of the analysis SRTM v4 (\citet{SRTM}) which is a 90m grid resolution DEM has been used. Other DEM might have been used to link the GlaThiDa with RGI (see section \ref{GlaRGI})

\subsection{GlaThiDa}
The glacier Thickness Database (\citet{GlaThiDa2014}) is a set of data which attempts to put together all the available ice thickness measurements from glaciers and ice caps all around the globe. The database is composed of three different tables. The first one contains an overview of the database, the second one thickness data from maps or digital elevation models (DEM), and the third one, the one used in this thesis, contains the actual point measurements with the thickness data. The measurements were taken with different methods such as terrestrial and airborne radio-echo sounding, ground penetrating radar, direct drilling and other methods. The first database was released in 2014. In 2016 a second version 2.0 and its correction 2.1 were released, and in 2019 version 3.0 and 3.01 were finally released. All the analysis in this work have been done using version 2.1 as version 3.0 was released after a big part of the analysis was already done. According to the \href{https://github.com/ezwelty/glathida/blob/master/CHANGELOG.md}{change log} however most of the changes have been done to the structure of the data base. There were however some measurements additions to the database. The addition of some measurements for some glaciers in Switzerland in particular would be relevant for this thesis, given that most of the analysis done in this work has been conducted over the alpine glaciers. 
%Use subsections to structure your thesis. The first and second component of the
%momentum equation is shown in equation (\ref{2equ:1}) and (\ref{2equ:2}),
%respectively. Together with (\ref{2equ:3}) they form the set of shallow-water
%equations implemented in a numerical model.

\subsection{RGI}
The Randolph Glacier Inventory (RGI) (\citet{RGI2014}) is a global database of outlines of glaciers, excluding ice sheets. The inventory has been compiled from satellite imagery collected from 1999 onward. Most of the outlines don't express a specific picture in time of the glacier but are a compound of different images of each glacier due to images obstructions such as cloud covers or satellite orbits. The first version of this inventory was released in 2012 and the latest version, RGI Version 6.0 was released in 2017. In this latest release the database comprises more than 220,000 glacier outlines. This is the version used for all the analysis in this thesis.

\subsection{Linking GlaThiDa and RGI}\label{GlaRGI}
linking them using OGGM

\subsection{Some stats about GlaThiDa}
\href{https://oggm.org/2019/03/21/GlaThiDa-statistics/}{GlaThiDa Statistics} about glaciers distributions and so on 


\section{Choosing Features}\label{features}

\subsection{Putting together the features}
How did we get the features for training: OGGM.

\subsection{Which features did we choose}
The features we chose and why.

\section{Machine Learning Models}\label{ML}
What they are and how they work on the high level

\subsection{Tuning parameters}
How did we decide which parameters to use (maybe we can just leave it for the appendix)

\subsection{SVM}
Explain Support Vector Machine (do i need to write down the specific mathematics?)

\subsection{Random Forest}
Random Forest

\subsection{Linear Regression}
Linear Regression

\section{Training method}\label{training}
use sklearn train\_test\_split 20 times

\section{Scoring method}\label{scoring}
metrics to compare the goodness of models

\section{Features Importance}\label{featuresimp}
what is feature importance.

\subsection{Shuffle}
Shuffle

\subsection{partial dependence plot}
partial dependence plot 
%\subsubsection{Subsubsection}
%You can also use ``subsubsections''. However, they do not carry a separate
%heading number and they do not appear in the Table of Contents.

%\subsection{Equation}
%As an example for the \verb|equation| environment, I show the equations used in
%the numerical shallow-water model (SWM) developed by
%\citet{scha93Aag,scha93Bag}:
%% ---- equation 1:
%\begin{equation}
%\frac{D\hat{u}}{D\hat{t}}+\frac{\partial(\hat{h}+\hat{H})}
%                               {\partial\hat{x}}=0,
%\label{2equ:1}
%\end{equation}
%% ---- equation 2:
%\begin{equation}
%\frac{D\hat{v}}{D\hat{t}}+\frac{\partial(\hat{h}+\hat{H})}
%                               {\partial\hat{y}}=0,
%\label{2equ:2}
%\end{equation}
%% ---- equation 3:
%\begin{equation}
%\frac{\partial\hat{H}}{\partial\hat{t}}+\frac{\partial(\hat{u}\hat{H})}
%                                             {\partial\hat{x}}
%                                       +\frac{\partial(\hat{v}\hat{H})}
%                                             {\partial\hat{y}}
%                                                =0,
%\label{2equ:3}
%\end{equation}
%with the non-dimensional variables (henceforth generally labelled with hats)
%$\hat{u}$ and $\hat{v}$ as the two horizontal velocity components, $\hat{H}$ and
%$\hat{h}$ as fluid layer depth and terrain height, respectively,
%$\hat{Z}=\hat{h}+\hat{H}$ as fluid layer height, and $\hat{t}$ as time.
%Equations~(\ref{2equ:1})--(\ref{2equ:3}) are non-dimensionalized with the
%following scales: a typical length $L$ for the horizontal length scale, the
%initial far-upstream depth of the fluid layer $H_{\infty}$ (with $h_{\infty}=0$)
%for the vertical length scale, the phase speed of linear gravity waves
%$\sqrt{g^*H_{\infty}}$ for the velocity scale, and the time scale
%$L/\sqrt{g^*H_{\infty}}$.
