\chapter*{Abstract}
\addcontentsline{toc}{chapter}{Abstract}
\thispagestyle{plain}
Melting glaciers play an extremely important role in the context of climate change. Correctly estimating their volumes is essential to accurately predict future in sea level and water availability. To compute the volume of a glacier, an estimate of its thickness distribution is needed. Ice thickness measurements are sparse and spatially incomplete. Numerical and statistical modeling therefore is necessary to estimate the full ice thickness distribution. Recently the Ice Thickness Models Intercomparison eXperiment (ITMIX) evaluated the accuracy of 17 ice thickness estimation models. 15 of those models numerically relay on physical processes and numerical modeling, in order to estimate the ice thickness distribution. Recently in fact, statistical models based on ice thickness observations, have not been used extensively by the glaciology community. The recent publication of the Glacier Thickness Database (GlaThiDa), a data-set aiming at collecting all glacier ice thickness observations worldwide, and the advancement in the field of machine learning, opened up to the possibility of a glacier ice thickness estimation model based on machine learning algorithm.

In this thesis three different machine learning algorithm have been trained using ice thickness observations from the GlaThiDa for glaciers in the Alps. Linear regression, random forest regression and support vector regression have been used to create models to estimate glaciers ice thickness. Part of the ice thickness observations have been used to train the models and part to test their accuracy. With this cross-validation method 20 different samples have been created to train and test the models. The highest $R^2$ coefficient has been registered by the support vector machine, with an average $R^2=0.55$ across the different samples. 

The algorithms have then been trained using all the available ice thickness observations for glaciers in the alps, i.e. not performing cross-validation. The highest $R^2=0.57$ has been registered for the random forest regression, which could however be subject to bias towards the data used for training it. The three trained models from this last experiment have been used to estimate the total volume of glaciers in the alps. The results have been compared to the findings from the third ITMIX experiment, which estimated the volume of glaciers globally using five established physically based models. The linear regression model predicted a total volume 9.8\% larger than the reference estimate. The random forest regression a volume 22.1\% lower and the support vector machine a volume 25.1\% lower. The last two models have been found unable to correctly estimate ice thicknesses for large glaciers in the alps.

The results indicate that the machine learning algorithms trained in this thesis, are not fit to estimate glaciers ice thickness distribution and their total volume of ice, for the alpine region. Training the models with data from other regions and measuring their performances would be the next step to take.  Training with more input data, could also result in a different outcome. The availability of more ice thickness observations, especially for the largest glaciers, which will hopefully increase in the future years, could result in the biggest improvement in the models accuracy. An accurate glacier model based on machine learning, could then be used to extend the ensemble relaying on the physically based model from the last ITMIX, and improve the accuracy of the ensemble. 
%The abstract is a short summary of the thesis. It announces in
%a brief and concise way the scientific goals, methods, and most important
%results. The chapter ``conclusions'' is not equivalent to the abstract!
%Nevertheless, the abstract may contain concluding remarks. The abstract
%should not be discursive. Hence, it cannot summarize all aspects of the thesis
%in very detail. Nothing should appear in an abstract that is not also
%covered in the body of the thesis itself. Hence, the abstract should be the
%last part of the thesis to be compiled by the author.
%
%A good abstract has the following properties: \emph{Comprehensive:} All major
%parts of the main text must also appear in the abstract. \emph{Precise:}
%Results, interpretations, and opinions must not differ from the ones in the main
%text. Avoid even subtle shifts in emphasis. \emph{Objective:} It may contain
%evaluative components, but it must not seem judgemental, even if the thesis
%topic raises controversial issues. \emph{Concise:} It should only contain the
%most important results. It should not exceed 300--500 words or about one page.
%\emph{Intelligible:} It should only contain widely-used terms. It should
%not contain equations and citations. Try to avoid symbols and acronyms (or at
%least explain them). \emph{Informative:} The reader should be able to quickly
%evaluate, whether or not the thesis is relevant for his/her work.
%
%An Example: The objective was to determine whether \dots (\emph{question/goal}).
%For this purpose, \dots was \dots (\emph{methodology}). It was found that \dots
%(\emph{results}). The results demonstrate that \dots (\emph{answer}).
