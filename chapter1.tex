\chapter{Introduction}\label{chap1}
\thispagestyle{plain}

% ====SECTION 1 ================================================================
\section{Motivation}\label{motivation}
Estimating the volume of glaciers is becoming an increasingly important topic, due to the role this plays in climate change. Melting glaciers have in fact an impact in sea level rise \citep{Zemp2017}, availability of fresh drinking water \citep{Kaser2010}) and even energy supplied by hydro electric power plants\citep{Terrier2011}. A reliable estimate of glaciers volume is key in order to properly compute the entity of these and other problems.

Computing the volume of glaciers is a trivial task once in possession of the data needed for it: the glacier surface and its distributed ice thickness. These data however are not easily obtainable due to the amount of effort needed to collect them.

A very big effort in this regard has been carried out in the creation of the Randolph Glacier Inventory (RGI) \citep{RGI2014}, which is a data-set containing a globally complete inventory of glaciers and their surface shapes. From these surface shapes it is easy to compute the surface area of the glaciers.

Ice thickness observations are collected in the Glacier Thickness Database (GlaThiDa) \citep{GlaThiDa2014}, which is the first global effort in collecting all the available ice thickness data in a single database. This These observations however, when compared with the glaciers contained in the RGI, are very sparse (see \ref{glathida}). Out of the over 215,000 glaciers included in RGI, only 771 have thickness observations and, for most glaciers, there are not enough measurements to have a complete thickness distribution along the glacier surface. 

Modeling glaciers is the only alternative to obtain an estimate of their volumes and is where a big effort from the scientific glaciology community is directed.
  
%The chapter Introduction leads the reader into the subject matter
%of the thesis. It is sometimes called
%Statement/Formulation/Definition/Presentation of the Problem. It may start with
%a so-called Motivation. It also contains the State of Knowledge or State
%of Research which is based on a literature survey (see section \ref{1sec:2}).
%Further, it contains the Scientific Questions and/or the Goals that are
%addressed in the main part of the thesis (see section \ref{1sec:3}). Finally,
%it provides an Outline of the science thesis (see end of section \ref{1sec:3}). 


% ==== SECTION 2 ===============================================================
\section{State of Research}\label{research}
Over the years a number of different approaches have been used to model glaciers and estimate their volumes.
The simplest ones only attempt to compute these volumes from the relationship between the glacier surface and its volume as first stated by \citet{bahr1997}. Using the Buckingham Pi
theorem they found that the relationship between the surface area $S$ and the volume of a glacier $V$ is $V \propto S^{\gamma}$ and they estimated $\gamma=1.375$ for these glaciers and $\gamma=1.25$ for ice caps. Similar approaches also take into consideration other other parameters such as glacier length, width, elevation range \citep{Grinsted2013}, and surface slope. These type of methods are called ``scaling approaches'' and they yield to a volume estimate for each glacier but not to the distributed ice thickness. Due to the small requirements for their implementations they are very effective at computing the global volume of glaciers.

A more sophisticated approach to estimate the volume of glaciers is to reconstruct its three dimensional shape with a model, and thus calculate its volume. Traditionally this is done by making physical assumptions about the glacier dynamics and solving equations deriving from these assumptions to estimate the glacier shape. Numerical methods are then used to obtain a discretization of these equations and solve them with computers. The analytical solutions for the equations are in fact often unknown, hence their solutions are calculated by computers on discrete grid-maps. This set of numerically solved equations are what compose the model. 
For glaciers this means reconstructing their bed shapes from their surface values, hence inverting the bed shape from the surface measurements using the constructed model.
 

%The equations needed to create the glacier map depend on parameters which are often unknown. This creates the inverse problem of having to estimate these parameters from observed data. In the case of estimating the glacier ice thickness to compute its volume, the problem would then be to tune the model parameters of the numerically solved equations, to best reproduce the observed ice thickness.  
%These equations depend on parameters which need to be tuned in order for the equations to replicate the available observations. The tuning of these parameters is known as the inverse problem which needs to be solved in order for the  In the case of the    This can be done starting from te known values of the glaciers surface such as the glacier shape, available in the RGI, and the topography of the glacier, obtainable from a digital elevation model. This leads to a grid-map containing the glacier surface shape and the elevation of each grid point. Using these and potentially other values such as the glacier climate, one could create an algorithm based on physical assumptions or statistical data, to estimate the bed shape of the glacier. able to numerically compute the glacier bed shape using either a physical or statistical numerical   and extrapolate the volume from the results of this simulation. Usually physical assumptions must be made to be able to reproduce the glacier bed from its surface ice distribution.
First steps in this direction were made by \citet{Nye1965} in the study of  idealized glaciers. He pointed out that the distributed ice thickness could be inferred by the glacier surface slope using estimates of basal shear stress. From there many others developed this idea, such as \citet{haeberli1995} who made the assumption of a known constant basal shear stress, parameterized from the glacier elevation range, to estimate basic glaciological characteristics for glaciers in the Alps. 

From these basic concepts, many more approaches have emerged in the years after, taking for example the mass conservation into account \citet{rasmussen_1988}, basal slipperiness together with bedrock topography \citet{Gudmundsson2001}, information such as mass balance and surface velocities (e.g. \citealt{gantayat2014}, \citealt{brinkerhoff2016}) and even more complex ones using forward models of ice flow \citet{vanPelt2013}. Also some non physical methods such as artificial neural networks \citet{Clarke2009}, have been adopted all to tackle the problem of estimating glaciers ice thickness. 

All these achievements have been possible thanks to the advancement in research in the glaciological field and to the steadily increase in computational power and its availability. Models which could only be numerically computed for single glaciers, started expanding their range of possibilities, and \citet{Huss2012} released the first glacier model used to model all known glaciers and ice caps worldwide. Since then many more global scale glacier models able to predict ice thickness distribution have been released. 

Meanwhile the GlaThiDa database was being compiled. This is a a collection of in-situ observations of glaciers ice thickness, created by joining together in a single database all the available ice thickness observations from worldwide glaciers. In particular it contains a table listing point-wise ice thickness measurements and their geographical coordinates. Even though these data are far from being enough to obtain ice volume estimates for all glaciers in the world, as mentioned in section \ref{motivation}, they are very useful for assessing glacier models performances. 

The compilation of GlaThiDa together with the increase number of models being adopted by the glaciology community led to the first Ice Thickness Models Intercomparison eXperiment (ITMIX) \citep{Farinotti2017}, which is the first attempt to put together all those models and asses their accuracy on the base of known ice thickness observations available in the GlaThiDa, as well as comparing the models performances with each other. 

The experiment focused on 17 glacier models which were given the task to estimate the ice thickness distribution of at least some of 15 glaciers, 3 ice caps and 3 synthetically grown glaciers. In order to predict the ice thickness the models were not allowed to use ice thickness observations for calibration. The output of the models was then compared with data available from the GlaThiDa and with the other models. A composite solution from all the participating models achieved a local ice thickness difference in the order of $10 \pm 24\%$ of the mean ice thickness. In general however the spread of the solutions obtained by the single models often exceeded the local ice thickness, concluding that an ensemble approach could mitigate errors and would reduce the random nature of single models output.  

The models participating in the first ITMIX were divided into categories depending on the different approaches used to model the ice thickness. The categories are: (1) those taking ice thickness inversion approaches as a minimization problem, (2) those based on mass conservation,  (3) approaches depending on a basal shear stress parametrization, (4) approaches based on surface velocities and (5) other approaches. 15 out of the 17 explored models in the ITMIX, belong to the first 4 categories, which are all categories for methods relaying on physical assumptions to model numerically some of the physical processes happening in a glacier and predict the ice thickness from there. Only 2 models, belonging to category (5) are models not based on physical assumptions and methods, but rather on statistical methods. Of those two the one from \citet{Clarke2009} is based on artificial neural networks, a branch of machine learning. It uses the input data from artificially grown glaciers to make predictions about real glaciers ice thickness, learning from the input data of the artificially grown ones. At the time of Clarke's publication in fact the GalThiDa had not yet been compiled, and there were not enough ice thickness observations available, to train a machine learning model from real glacier data.

Machine learning is a branch of computer science which uses different algorithms to give a computer the ability to learn from a set of samples. The concept of artificial intelligence, which is the super field of machine learning, was first proposed by \citet{Turing1950}.

Many different machine learning algorithms have been developed since 1950 and some even before. Although the algorithms have been around for a long period of time, they only saw wide spread application starting from the late 1990, as computation power and resources largely improved. Since then however the field has seen a huge development and many of its applications have become part of our day to day life. Things such as voice recognition, chat bots, autonomous driving and online products suggestions, are all a products of machine learning algorithms being developed and trained to solve these specific problems.

The basic idea behind machine learning is to allow a computer program to generate a model capable of solving a task, without explicit instructions on how to solve it. In the case of the glacier thickness it would mean to feed the algorithm a data sample including observed glacier ice thicknesses, to learn form possible patterns within this sample. After the learning process is finished one could use the trained model (as algorithms which underwent the learning process are called) to make predictions about glaciers with no thickness observations, or to fill the holes for those with too few ones.

Machine learning however has yet to make a big impact in the glaciological community and most of the models used in the ITMIX, and the ones which achieved better accuracy, are physically based models. In a later experiment, \citet{Farinotti2019} developed an ensemble model, using 5 of the physically based models form the first ITMIX, to better predict ice thickness distributions of all glaciers in the world. The results from this experiment can be used as good benchmark to compare models based on machine learning algorithms.

The importance glaciers play in climate change and climate modeling, the availability of ice thickness observations, the excitement behind machine learning in almost every possible field, and the lack of previous work done in this sense by the glaciological modeling community, are the driving motivation factors for this thesis.

%Based on the literature survey, the writer draws a picture of the existing
%knowledge in a specific field and points to open questions. Hence, after this
%survey the Introduction will ultimately culminate in the formulation of specific
%scientific questions/goals addressed in the thesis (see section \ref{1sec:3}).

%To cite a certain source (e.g., a paper) use the citation commands
%\verb|\citet| and \verb|\citep| of the \verb|natbib|
%package.\footnote{
%\url{http://www.ctan.org/tex-archive/macros/latex/contrib/natbib/natnotes.pdf}}
%Together with the bibliography style \verb|ametsoc.bst|, which is
%included in the \LaTeX{} manuscript template for AMS journals\footnote{
%\url{http://www.ametsoc.org/PUBS/journals/AMS_Latex_V3.0.tar.gz}}, \verb|natbib|
%produces citations in the author-date format together with a list of
%references that fulfill the AMS citation standard.\footnote{
%\url{http://www.ametsoc.org/PUBS/journals/author_reference_guide.pdf}}
%
%As an example, you can cite papers like \citet{hann66Aag} and \citet{scha93Aag}
%which have to be specified in your BibTeX database file (in this case it is
%\verb|mybibfile.bib|). More than one article of the same author can be cited
%like here: \citet{hoin85Aag,hoin90Aag} studied foehn winds.
%
%You may want to split your review of the literature into several sections.
%Further, use paragraphs to structure your introduction. If you like to cite
%papers in brackets (\emph{passive citations}) you can do this
%as in the following sentence: Gap flows have been studied in the Strait
%of Gibraltar \citep{scor52Aag,dorm95Aag}, in the French Rh\^one Valley
%\citep{pett82Aag}, near Hokkaid\=o in Japan \citep{arak69Aag}, near
%Unimak Island in the Aleutian Chain \citep{pan-99Aag}, and in the Howe
%Sound of British Columbia \citep{jack94Aag,jack94Bag}. Citation of a
%Dissertation: The gap flow in the Wipp Valley has been studied by
%\citet{gohm03Aag}. Citation of a conference paper: \citet{gohm06Aag}
%investigated the boundary layer structure in the Inn Valley. Citation of an
%online document: The AMS provides a guideline for preparing citations and
%references \citep{ams-09Aag}.


% ==== SECTION 3 ===============================================================
\section{Goals and Outline}\label{goals}
In this thesis three different machine learning algorithms have been chosen to  create models able to predict glacier ice thickness. The algorithms have all been trained on the same data-set, the GlaThiDa which contains ice thickness observations for glaciers worldwide.

To train the algorithm a set of attributes (often called features), mainly geometrical, have been generated using the Open Global Glacier Model (OGGM) \citep{OGGM2019}. This attributes are all computable by the only mean of a digital elevation model for the considered glacier, and the outline of the glacier given by the RGI.

The accuracy of the models have been estimated comparing their predictions with the observations from the GlaThiDa as well as with themselves.

An analysis of the most influential attributes in making predictions has been carried out.

Finally the obtained models have been used to predict the volume of all the alpine glaciers and compare their results with those of the third ITMIX \citep{Farinotti2019}.


This thesis then tries to establish:
\begin{itemize}
\item[(1)] How accurate machine learning algorithms are at predicting glaciers ice thickness.
\item[(2)] How three different machine learning algorithms compare with each other in estimating glaciers ice thickness.
\item[(3)] Using those algorithm to estimate the total volume of glaciers in the alps, how these models compare to the model from the \cite{Farinotti2019} elaborated in the third ITMIX.
\end{itemize}

%The so-called SMART
%criteria\footnote{\url{http://en.wikipedia.org/wiki/SMART_criteria}} might be
%used as a guideline to define reasonable goals. Here, the acronym SMART
%describes the properties of ``good'' goals: \underline{s}pecific,
%\underline{m}easurable, \underline{a}ttainable, \underline{r}elevant,
%\underline{t}ime-bound.
%
%The introduction may also describe briefly the methodology chosen and the
%materials (e.g., data, instruments, etc.) used. However, a detailed description
%will follow in the main part (see chapter \ref{chap2}).
%
%Finally you should present an \emph{outline} of your science thesis. Explain
%what the reader will find in the following chapters. For example, chapter
%\ref{chap2} describes the methodology. The results are presented
%in chapter \ref{chap3}. A discussion is provided in chapter \ref{disc} and the
%conclusions are drawn in chapter \ref{concl}.
