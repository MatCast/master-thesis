\chapter{Introduction}\label{chap1}
\thispagestyle{plain}

% ====SECTION 1 ================================================================
\section{Motivation}\label{motivation}
Estimating the volume of glaciers is becoming an increasingly important topic, due to the role this plays in computing the impact of melting glaciers which is associated with climate change. Melting glaciers have in fact an impact in sea level rise(\citet{Zemp2017}), availability of fresh drinking water (\citet{Kaser2010}) and even in energy supplied by hydro electric power plants(\citet{Terrier2011}). A reliable estimate of glaciers volume is key in order to properly compute the entity of those problems.

Computing the volume of glaciers is a trivial task once in possession of the data needed for it: the glacier surface and its distributed ice thickness. These data however are not easily obtainable due to the amount of effort needed to collect them. A very big effort in this regard has been done with the Randolph Glacier Inventory (RGI) (\citet{RGI2014}), which is a data set containing a globally complete inventory of glaciers with their surface shapes. The data on ice thicknesses for these glaciers however are much more lackluster(see \ref{featuresimp}). Of the over 200,000 glaciers included in the RGI only 771 have thickness data, and in most of the cases not enough to have a complete thickness distribution along the glacier surface. These data are collected in the Glacier Thickness Database (GlaThiDa \citet{GlaThiDa2014}).

To obtain an estimate of volume for glaciers lacking thickness measurements (the vast majority of the global glaciers), the only possible alternative is to model them.
  
%The chapter Introduction leads the reader into the subject matter
%of the thesis. It is sometimes called
%Statement/Formulation/Definition/Presentation of the Problem. It may start with
%a so-called Motivation. It also contains the State of Knowledge or State
%of Research which is based on a literature survey (see section \ref{1sec:2}).
%Further, it contains the Scientific Questions and/or the Goals that are
%addressed in the main part of the thesis (see section \ref{1sec:3}). Finally,
%it provides an Outline of the science thesis (see end of section \ref{1sec:3}).

\begin{itemize}
	\item[(1)] Hype in machine learning (both in academic and business world).
	\item[(2)] Hype in estimating glacier ice thickness
	\item[(3)] Most models use physical based approaches. Use a statistical one.
\end{itemize}
 


% ==== SECTION 2 ===============================================================
\section{State of Research}\label{research}
In the years a number of different approaches have been used to model the glaciers and estimate their volumes.
The most simple ones only attempt to compute the volume of the glacier from the relationship between the glaciers surface and it's volume as first stated by \citet{bahr1997}. Using the Buckingham Pi
theorem they found that the relationship between the area $S$ and the volume of a glacier $V$ is $V \propto S^{\gamma}$ for valley glaciers, and they estimated $\gamma=1.375$ for these glaciers and $\gamma=1.25$ for ice caps. Simliar approaches try also take in consideration other other parameters such as glacier length, width and elevation range (\cite{Grinsted2013}), and surface slope. These type of methods are called ``scaling approaches'' and they yield to a volume estimate for each glacier but not to the distributed ice thickness. Due to the small requirements they are very effective at computing the global volume of glaciers.
Modeling the distributed ice thickness of glaciers has been the subject of many studies since the 
%Based on the literature survey, the writer draws a picture of the existing
%knowledge in a specific field and points to open questions. Hence, after this
%survey the Introduction will ultimately culminate in the formulation of specific
%scientific questions/goals addressed in the thesis (see section \ref{1sec:3}).

%To cite a certain source (e.g., a paper) use the citation commands
%\verb|\citet| and \verb|\citep| of the \verb|natbib|
%package.\footnote{
%\url{http://www.ctan.org/tex-archive/macros/latex/contrib/natbib/natnotes.pdf}}
%Together with the bibliography style \verb|ametsoc.bst|, which is
%included in the \LaTeX{} manuscript template for AMS journals\footnote{
%\url{http://www.ametsoc.org/PUBS/journals/AMS_Latex_V3.0.tar.gz}}, \verb|natbib|
%produces citations in the author-date format together with a list of
%references that fulfill the AMS citation standard.\footnote{
%\url{http://www.ametsoc.org/PUBS/journals/author_reference_guide.pdf}}
%
%As an example, you can cite papers like \citet{hann66Aag} and \citet{scha93Aag}
%which have to be specified in your BibTeX database file (in this case it is
%\verb|mybibfile.bib|). More than one article of the same author can be cited
%like here: \citet{hoin85Aag,hoin90Aag} studied foehn winds.
%
%You may want to split your review of the literature into several sections.
%Further, use paragraphs to structure your introduction. If you like to cite
%papers in brackets (\emph{passive citations}) you can do this
%as in the following sentence: Gap flows have been studied in the Strait
%of Gibraltar \citep{scor52Aag,dorm95Aag}, in the French Rh\^one Valley
%\citep{pett82Aag}, near Hokkaid\=o in Japan \citep{arak69Aag}, near
%Unimak Island in the Aleutian Chain \citep{pan-99Aag}, and in the Howe
%Sound of British Columbia \citep{jack94Aag,jack94Bag}. Citation of a
%Dissertation: The gap flow in the Wipp Valley has been studied by
%\citet{gohm03Aag}. Citation of a conference paper: \citet{gohm06Aag}
%investigated the boundary layer structure in the Inn Valley. Citation of an
%online document: The AMS provides a guideline for preparing citations and
%references \citep{ams-09Aag}.

Background of the literature: GlaThiDa, ITMIX (\citet{Farinotti2017}), etc 


% ==== SECTION 3 ===============================================================
\section{Goals and Outline}\label{goals}

%After the literature survey the Introduction will ultimately culminate in the
%formulation of specific scientific questions, aims or \emph{goals}. Hence, near
%the end of the Introduction there will often appear sentences like:\\[1ex]
%\dots The goal of the investigation thus became trying to find out if \dots\\
%\dots For this reason it appeared reasonable to attempt \dots\\
%\dots It therefore became necessary to clarify whether \dots\\ 
%
%Instead of describing the goals in one paragraph, you may want to structure them
%with the \verb|itemize| command:
\begin{itemize}
\item[(1)] How well can Machine Learning Algorithm predict glaciers ice thickness.
\item[(2)] How do 3 different machine learning algorithm compare with each other in estimating glaciers ice thickness.
\item[(3)] If we use those algorithm to estimate the total volume of glaciers in the alps, how do these model compare to some of the physical based ones for this region.
\end{itemize}

%The so-called SMART
%criteria\footnote{\url{http://en.wikipedia.org/wiki/SMART_criteria}} might be
%used as a guideline to define reasonable goals. Here, the acronym SMART
%describes the properties of ``good'' goals: \underline{s}pecific,
%\underline{m}easurable, \underline{a}ttainable, \underline{r}elevant,
%\underline{t}ime-bound.
%
%The introduction may also describe briefly the methodology chosen and the
%materials (e.g., data, instruments, etc.) used. However, a detailed description
%will follow in the main part (see chapter \ref{chap2}).
%
%Finally you should present an \emph{outline} of your science thesis. Explain
%what the reader will find in the following chapters. For example, chapter
%\ref{chap2} describes the methodology. The results are presented
%in chapter \ref{chap3}. A discussion is provided in chapter \ref{disc} and the
%conclusions are drawn in chapter \ref{concl}.
