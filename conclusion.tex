\chapter{Conclusions}\label{concl}
\thispagestyle{plain}

In this thesis three machine learning algorithms have been used to create models aimed at predicting glaciers ice thickness.

In order to do so, the GlaThiDa, a database with ice thickness observations of glaciers all over the world has been linked to the RGI, a database of glaciers outlines which include over 215,000 glaciers and represents all glaciers and ice caps in the globe. Of all the observations in GlaThiDa, 96.6\% have been associated with glaciers in the RGI, summing up to a total of only 771 glaciers worldwide with ice thickness observations. In many glaciers only few observations have been reported in the GlaThiDa, and some regions completely lack ice thickness measurements.

In order to create a data set to train the machine learning models, the Open Global Glacier Model has been used. This model provided the tools to merge together the relevant digital elevation models for each glacier, and compute gridded geometrical features such as: the topography, the distance from the border of the glacier, the slope angle and the altitude dependent linear mass balance.

With this data three different machine learning models, linear regression, random forest regression and support vector regression, have been trained using only observations available for the alpine region. To avoid over-fitting the models, and to be able to assess their performances, each model has been trained with 20 different sub-samples, each containing 75\% of data points available. The $R^2$ coefficient, also called score, calculated over the resulting 25\% of the data, has been chosen as a value to assess the performances of each model. The best performing model on average was the support vector regression with an average $R^2 = 0.55$. The spread of the score between the models trained with the different sub-samples has been found particularly high with much lower scores for some specific sub-samples. This is true for the three different machine learning algorithms used, as all of them showed drops in score for samples 5, 14 and 19. This behavior seems explainable by the fact that the sub-sample left out for training in these 3 cases was particularly hard to predict.
The model achieving the highest score when training them over the whole sample has been the random forest regression with a score of 0.57.

After the models have been trained the relevance of each feature in the models predictions has been analyzed. The mass balance was the most relevant feature for both the random forest and support vector regression, but for the linear regression it was the slope angle. Those two features however proved to be relevant for the predictions of all the models. Altitude and distance from the border seem to be much less important for the predictions of both the random forest and the linear regression model, while still being relevant for the support vector regression.

The trained machine learning models have been used to estimate the ice thickness of each of the glaciers in the Alps. The results have been compared to the one obtained by \citet{Farinotti2019}. All the machine learning models have been found to predict very different volumes compared to the ones predicted in the ITMIX. The linear regression is the only model which predicted a total ice volume larger than the one from ITMIX by over 9\%. The random forest and support vector machine predicted a total volume respectively 22\% and 25\% lower than ITMIX. These two models in fact seem to underestimate the ice thickness for large glaciers, leading to low volume values for those glaciers and hence for the whole region. Larger glaciers are essential in estimating the total volume of ice but are also outliers in the per-volume distribution of glaciers in the alps. The machine learning algorithms seem then unfit to make predictions about these larger glaciers. This could be explained by the fact that the models were not able to learn the patterns of these large glaciers, due to the fact that they had no information about them in the training data-set. 

After all, the machine learning algorithms have not been very accurate in estimating the ice thickness using the features chosen as input values. This applies particularly to large glaciers, for which the models seem unable to replicate accurate ice thicknesses.
Further research could use different models and add more features to the input data.
The results obtained for the Alps could be different for other regions especially if those regions had ice thickness observations covering a wider spread of glacier dimensions (small, mid an large glaciers equally).
Given the importance of large glaciers in computing ice volumes, having more ice thickness measurements for those glaciers could improve the accuracy of the models at predicting their thicknesses, and therefore the ice volumes.
